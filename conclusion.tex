\section{Conclusion}

I our work, the not very explored research area of multi-instance clustering was investigated. Three methods for clustering of bags were introduced, of which one is unsupervised (CPC) and two are supervised (Triplet loss and Magnet loss). For each of the methods, the prior art it builds on was presented, along with its modification for the purposes of multi-instance clustering. All three of the methods were theoretically and experimentally evaluated and compared. The experiments were conducted first on publicly available datasets in a reproducible fashion. Following that, a corporate dataset of network security data was used as it is the primary application in mind for this work. 

Comparing the methods on the publicly available datasets shows the method based on contrastive predictive coding to perform the worst, with the other having no statistically significant difference between them. Similar results were also obtained on the corporate dataset of HTTP traffic, albeit none of the results were as good as anticipated. The method based on contrastive predictive coding performed poorly on both the publicly available datasets and the corporate one, however, the comparison might not be fair as the CPC method is unsupervised, whereas the other two can utilize labels on the training data, giving them a strong advantage.

Clearly, more research is needed. The most promising method, CPC, performed poorly and should be investigated more. A thorough theoretical investigation of the method including finding its local extrema might explain its low performance, however, it was judged to be outside the scope of this work. Combining the representative power of multi-instance learning with the versatility of the clustering algorithms remains an open problem.
